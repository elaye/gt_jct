%%!TEX encoding = UTF-8 Unicode

\documentclass[a4paper,12pt]{article}
\usepackage{geometry}
\usepackage{graphicx}
\usepackage{amsmath}
\usepackage{amssymb}
\usepackage{xspace}
\usepackage{tikz,ifthen,fullpage}
\usepackage{xcolor}
\usepackage{sistyle}
\usepackage{epstopdf}
\usepackage[frenchb]{babel}
\usepackage[utf8]{inputenc}
\usepackage[T1]{fontenc}
\geometry{dvips,a4paper,margin=1.5in}
\DeclareGraphicsRule{.tif}{png}{.png}{`convert #1 `dirname #1`/`basename #1 .tif`.png}

\title{Etude des trajectoires électroniques dans un guide par la méthode des éléments finis}
\author{Elie Génard, Félix Tora}
\date{Mai 2013}


\begin{document}
\maketitle

Un électron soumis à un champ électromagnétique est soumis à la force de Lorentz qui contient une composante due au champ électrique et une autre due au champ magnétique. Seule la partie électrique est capable de fournir du travail à l'électron et donc de l'accélérer. Un électron dans un champ électrique a donc une trajectoire qui dépend à la fois des conditions initiales (position et vitesse) et du champ électrique appliqué.

\section{Position du problème}

Le systéme étudié est un guide d'électrons. Des électrons sont émis depuis une source et sont "guidés" jusqu'à un détecteur dans un systéme coaxial. Le guide est constitué d'un fil porté au potentiel $V_0$ et d'un cylindre métallique mis à la masse. La symmétrie du problème nous motive à travailler dans un trièdre de coordonnées cylindriques d'axe le fil. Un schéma du dispositif est présenté sur la figure \ref{fig:schema}.

\begin{figure}[h]
\centering
\begin{tikzpicture}[scale=0.5]

%fil
\draw [very thick] (-5,0) -- (5,0);
\draw [-] (5,0) -- (5,4);
\path (5,4) node [anchor = south west] {$V_0$};

%deteceur source
\draw [thick, red] (-5.1,-1) -- (-5.1,1);
\draw [thick, red] (5.1,-1) -- (5.1,1);


%cylindre
\draw [-] (-10,-2) -- (10,-2);
\draw [-] (-10,2) -- (10,2);
\draw [-] (10,2) -- (10,4);
\draw [-] (9.625,4) -- (10.375,4);
\foreach \i in {1,...,7}                  %masse
   {\draw [-] (9.5 + 0.125*\i,4) -- (9.75 + 0.125*\i,4.25);
   }



%axes
\draw [dashed, ->] (-8,0) -- (8,0);
\draw [dashed, ->] (0,0) -- (0,3);
\path (8,0) node [anchor = west] {$z$};
\path ((0,3) node [anchor = south] {$r$};
\path (0,0) node [anchor = north] {$(0,0)$};

%electron
\fill (-3.5,1) circle (0.7mm);
\path (-3.5,1) node [anchor = south west] {$e^-$}; 

%longeurs
\draw [->] (-11,0) -- (-11,2);
\path (-11,1) node [anchor = east] {$R_2$};
\draw [<-] (-5,-3) -- (-0.4,-3);
\draw [->] (0.4,-3) -- (5,-3);
\path (0,-3) node {$L$}; 

\end{tikzpicture}
\caption{Schéma du guide d'électrons}
\label{fig:schema}
\end{figure}
Le rayon du cylindre est $R_2 = 3.5 \mathrm{cm}$, le rayon du fil est $R_1 = 0.15 \mathrm{cm}$. La longueur du fil (et donc du guide) est $L = 1 \mathrm{cm}$.

L'étude de la trajectoire des électrons dans le guide nécessite la connaissance du champ électrique dans ce dernier. Nous allons donc commencer par déterminer le champ électrique dans le guide avant de s'intéresser à notre problème qui concerne la trajectoire des électrons.

\section{Modèle simplifié du champ dans le guide}


On commence par envisager un modèle simple, qui consiste à dire que le tube est suffisamment grand pour pouvoir négliger les effets de bords (on aura l'occasion de revenir sur cette hypothése). On va donc considérer que le tube est infini. Le probléme présente alors des symétries intéressantes.


\subsection{Détermination du champ électrique et du potentiel dans le guide}

En premier lieu on déduit des invariances par translation selon l'axe $z$ et par rotation autour de ce même axe que le champ électrique $\mathbf{E}$ et le potentiel $V$ ne dépendent pas des coordonnées $z$ et $\theta$. Soit maintenant un point $M$ à l'intérieur du cylindre, il appartient à deux plans de symétrie du probléme (l'un contenant l'axe $z$ et l'autre perpendiculaire à ce même axe), on en déduit que le champ électrique au point $M$ est contenu dans ces deux plans. Le champ électrique dans le cylindre est donc de la forme
\begin{eqnarray}
\mathbf{E} &=& E(r) \mathbf{e_r}\\
V &=& V(r)
\end{eqnarray}

Un théorème de Gauss appliqué sur un cylindre d'axe $z$ nous donne pour le champ électrique et donc le pour le potentiel
\begin{eqnarray}
E(r) &=& \frac{\rho}{2 \pi \varepsilon_0 r}\\
V(r) &= &-\frac{\rho}{2 \pi \varepsilon_0} \ln(r) + C
\end{eqnarray}
où $\rho$ est la densité linéique de charges portée par le fil et $C$ une constante d'intégration à déterminer.

Les données du problème nous assurent que $V(R_2) - V(R_1) = V_0$, on en déduit ainsi l'expression de $\rho$ pour trouver le champ électrique et le potentiel
\begin{eqnarray}
\mathbf{E} &=& \frac{V_0}{\ln(\frac{R_2}{R_1})r} \mathbf{e_r}\\
V(r) &=& V_0 \frac{\ln(R_2) - \ln(r)}{\ln(R_2) - \ln(R_1)} 
\end{eqnarray}

\subsection{Equations de la trajectoire de l'électron}

Une fois une expression analytique du champ déterminée, il est relativement aisé d'écrire les équations du mouvement de l'électron dans le guide. Comme déjà suggéré, l'électron est soumis à la force de Lorentz et donc, dans notre systéme de coordonnées, le principe fondamental de la dynamique s'écrit de la maniére suivante :
\begin{eqnarray}
\ddot{r} - r \dot{\theta}^2 &=& \frac{e V_0}{m_e \ln(\frac{R_2}{R_1}) r} \\
r \ddot{\theta} + 2 \dot{r} \dot{\theta} &=& 0\\
\ddot{z} &=& 0
\label{eq:sys}
\end{eqnarray}
Ces équations écrites il nous faut choisir une méthode numérique pour calculer leur trajectoire.

\subsection{Calcul de la trajectoire de l'électron}
\label{subsection:lol}

Notre choix s'est d'abord porté sur une méthode de résolution d'équations différentielles à pas fixe : la méthode de Runge-Kutta d'ordre 4. Elle est relativement simple à implémenter et aussi relativement précise.

\paragraph{Système différentiel} Cette méthode permet de résoudre des équations différentielles du premier ordre, on doit donc mettre le système différentiel sous la forme d'une équation vectorielle du premier ordre. On va travailler avec les six variables suivantes : $r$, $\dot{r}$, $\theta$, $\dot{\theta}$, $z$ et $\dot{z}$, on note $Y$ le vecteur colonne ayant pour composantes les six variables dont il est question. La forme vectorielle du système d'équations différentielles \eqref{eq:sys} est donc

\[
Y^{\prime}
=
\begin{pmatrix}
\dot r\\
\ddot r\\
\dot \theta\\
\ddot \theta\\
\dot z\\
\ddot z
\end{pmatrix}
=
\begin{pmatrix}
\dot r\\
r \dot{\theta}^2\\
\dot{\theta}\\
- \frac{2 \dot r \dot \theta}{r}\\
\dot{z}\\
0
\end{pmatrix}
\]

\paragraph{Paramètres de calcul} Nous avons choisi un temps d'intégration de $\SI{3.10^{-8}}{s}$ (il est adapté à la vitesse initiale que l'on donne à l'électron). Ceci étant on choisit un pas d'intégration de $\SI{1.10^{-12}}{s^{-1}}$. Cette valeur est choisie de façon plus où moins arbitraire au début, elle présente certaines justifications cependant, la principale étant qu'avec un pas 10 fois plus petit le calcul ne termine pas dans un temps raisonnable.

Les conditions initiales que nous avons choisies initialement sont celles qui étaient données par l'énoncé, à savoir


\begin{eqnarray}
r_0 &=& \SI{0,03}{m} \label{ci:1}\\
\dot{r}_0 &=& \SI{0}{m.s^{-1}} \label{ci:2}\\
\theta_0 &=& \SI{0}{rad} \label{ci:3}\\
\dot{\theta}_0 &=& \SI{\frac{2,6.10^7}{r_0}}{rad.s^{-1}} \label{ci:4}\\
z_0 &=& -\SI{0,5}{m} \label{ci:5}\\
\dot{z}_0 &=& \SI{2,9.10^7}{m.s^{-1}}. \label{ci:6}
\end{eqnarray}




\paragraph{Présentation des résultats} Le tracé de la trajectoire obtenue est présenté sur la figure \ref{fig:t_3d}. On observe une trajectoire hélicoïdale qui correspond bien à l'idée que l'on peut se faire d'un guide d'électrons. Une fois émis, les électrons (qui possèdent les conditions initiales décrites précédemment), vont avoir une trajectoire hélicoïdale autour de l'axe $z$, à savoir le fil mis au potentiel $V_0$.

\begin{figure}[h]
\centering
\includegraphics[height=6cm]{images/t_3d.png}
\caption{Trajectoire d'un électron dans le potentiel sans effets de bords}
\label{fig:t_3d}
\end{figure}

Ce résultat est encourageant car il montre d'une certaine manière que notre méthode de résolution des équations différentielles est opérationnelle. Elle pourra être utilisée pour calculer des trajectoires dans des cas où le potentiel et le champs auront été déterminé autrement.


\section{Modèle simplifié tenant compte des effets de bord}
\subsection{Détermination numérique du potentiel}
\paragraph{Présentation de l'équation à résoudre}
On suppose maintenant que le dispositif n'est plus de longueur infinie, ce qui fait apparaître des effets de bord et nous empèche donc de résoudre analytiquement l'équation de Laplace ($\Delta V=0$). Cependant on peut tout de même simplifier le problème en admettant que le potentiel est à symétrie cylindrique, c'est à dire que $V$ est invariant selon $\theta$. On doit donc résoudre numériquement l'équation suivante:

\begin{equation}
\Delta V(r,z)=0
\end{equation}
Soit:
\begin{equation}
\frac{\partial^2 V}{\partial r^2}+\frac{1}{r}\frac{\partial V}{ \partial r}+\frac{\partial^2 V}{\partial z^2}=0
\end{equation}

\paragraph{Limites du modèle}
On remarquera que dans ce modèle on ne prend pas en compte le fil d'arrivée de potentiel qui en réalité déforme le potentiel à l'intérieur du tube. On ne prend pas non plus en compte le fil qui connecte le tube à la masse car nous travaillons à l'intérieur du cylindre et il n'a donc aucune influence.

\paragraph{Modèle numérique}
Afin de résoudre numériquement l'équation de Laplace, on utilise la méthode des éléments finis dans l'approximation de Galerkin qui consiste à pondérer l'équation aux dérivées partielles par une fonction test $\alpha_i$ à support réduit sur chacun des $N$ noeuds du maillage.

\paragraph{Formulation faible}
En projetant l'équation de Laplace sur $\alpha_i$ on obtient, en notant $D$ le volume du cylindre et $\partial D$ sa frontière

\begin{equation}
\int_{D} \alpha_i \Delta V d^3r = 0,
\end{equation}
or
\begin{equation}
div(\beta_i \nabla V) = \nabla \alpha_i \cdot \nabla V + \alpha_i \Delta V,
\end{equation}
donc
\begin{equation}
\int_{D} \left( div(\alpha_i \nabla V) - \nabla\alpha_i \cdot \nabla V \right) d^3r = 0.
\end{equation}
D'après le théorème de Green-Ostrogradski on obtient
\begin{equation}
\oint_{\partial D} \alpha_i \nabla V \cdot d\vec{S} - \int_{D} \nabla \alpha_i \cdot \nabla V d^3 r = 0
\end{equation}

\paragraph{Conditions aux limites}
Les conditions aux limites à appliquer sur le fil et sur le tube sont des conditions de Dirichlet puisque le potentiel du fil est fixé à $V_0=\SI{20000}{V}$ et que le tube est relié à la masse. Ces conditions nous donnent



\begin{eqnarray}
V(0,z) &=& V_0 \\
V(R_2,z) &=& 0
\end{eqnarray}


De plus, puisque l'on considère un cylindre ouvert, on va appliquer des conditions de Neumann à ses extrémités. Les extrémités du tube étant suffisamment éloignés des extrémités du fil, on peut considérer que le potentiel est constant, ce qui revient à dire que

\begin{equation}
\frac{\partial V}{\partial \vec{n} } = \nabla V \cdot \vec{n} = 0
\end{equation}
avec $\vec{n}$ la normale à la surface $\partial D$
La formulation faible devient donc

\begin{equation}
\int_{D} \nabla \alpha_i \cdot \nabla V d^3 r = 0
\end{equation}

\paragraph{Discrétisation}
On discrétise maintenant l'équation en décomposant les intégrales sur les éléments du maillage. On remplace le potentiel $V$ par son interpolée $V = \sum_{j} \alpha_j V_j$. On somme ensuite sur tous les éléments du maillage ce qui donne

\begin{equation}
\sum_{e \in NBE} \sum_{j \in e} \int_{D} V_j \nabla \alpha_i \nabla \alpha_j d^3 r = 0
\end{equation}

On obtient alors une équation matricielle du type $A 
\left( \begin{array}{c}
\vdots \\
V_j \\
\vdots \\
\end{array} \right) 
 = 0$ avec

\begin{equation}
 a_{ij} = \int_{D} \nabla \alpha_i \nabla \alpha_j d^3 r 
\end{equation}

\medskip

Les intégrales sont calculées en utilisant la méthode de Gauss ce qui donne pour un élément :

\begin{equation}
\sum_{k=1}^{NPI} \alpha_i (k) \alpha_j (k) detJ(k) w_k
\end{equation}
avec $detJ(k)$ le jacobien associé à la transformation de l'élément normalisé à l'élément réel et $w_k$ le poids de Gauss associé au point $k$. On complète ensuite la fonction \verb|integrales.m| qui va calculer les éléments de la matrice A. On remarquera le facteur $2 \pi y$ ajouté dans le calcul des intégrales qui vient du fait que notre problème est axisymmétrique.

%\lstinputlisting[language=Matlab]{src/integrales.m}


%e=fem.elt(ne);
%NBN=e.NBN;

%AE=zeros(NBN,NBN);
%BE=zeros(NBN, 1);

%eps0=1/(36*pi*1e9);

%switch (e.TYP)
%   case 1
%      [gauss]=polynomes_S2(fem, ne);
%      nrg=e.NRG;
%      Dn=fem.equ.Dn(nrg);
%      sigma =fem.equ.sigma (nrg);
%      
%      NPI=gauss.NPI;
%      pds=gauss.pds;
%      detJ=gauss.detJ;

%      for npi=1:NPI 
%         for ie=1:NBN 
%            alphai = gauss.alpha(ie, npi);
%            
%            BE(ie) =  0;  
%                
%            for je=1:NBN
%               alphaj = gauss.alpha(je,npi);
%               AE(ie,je) = 0;
%            end;
%         end;
%      end;
%                
%   case 2
%      [gauss]=polynomes_T3(fem, ne);

%      nrg=e.NRG;
%      eps=eps0*fem.equ.eps(nrg);
%      rho=fem.equ.rho(nrg);  

%      NPI=gauss.NPI;
%      pds=gauss.pds;
%      detJ=gauss.detJ;
%      yg=gauss.y;

%      for npi=1:NPI                                 
%         for ie=1:NBN 
%            alphai = gauss.alpha(ie, npi);
%            dalphai_dx = gauss.dalpha_dx(ie, npi);
%            dalphai_dy = gauss.dalpha_dy(ie, npi);
%                         
%            BE(ie) =  0;
\begin{figure}[h]
\begin{verbatim}
% Boucle sur chaque noeud de l'élément  
for je=1:NBN
   alphaj = gauss.alpha(je, npi);
   dalphaj_dx = gauss.dalpha_dx(je, npi);
   dalphaj_dy = gauss.dalpha_dy(je, npi);
   gradai_gradaj=dalphai_dx*dalphaj_dx+dalphai_dy*dalphaj_dy;
   AE(ie,je) = AE(ie,je)
   				+ 2*pi*gradai_gradaj*pds(npi)*detJ(npi)*yg(npi);
end;
\end{verbatim}
\caption{Extrait de \emph{integrales.m}}
\end{figure}
%         end;
%      end;    
%   end;

La fonction \verb|assemblage.m| reconstitue la matrice A. Les conditions de Dirichlet sont appliquées, par la fonction \verb|conditions.m|, à notre système en utilisant les fichiers \verb|.pro|. Une fois les calculs effectués, \verb|solution.m| s'occupe d'afficher les résultats de la simulation.

%...

\paragraph{Maillage}
On utilise le maillage de la Figure \ref{f mesh} pour la simulation. Le maillage est volontairement plus fin dans la zone proche du fil car c'est dans cette zone que l'on devrait logiquement observer les plus grandes variations du potentiel.
\begin{figure}[h]
\centering
\includegraphics[width=1\textwidth,height=0.25 \textwidth]{images/mesh}
\caption{Maillage utilisé pour la simulation}
\label{f mesh}
\end{figure}

\paragraph{Résultats de simulation}
Après simulation, nous observons un profil de potentiel donné par la Figure \ref{f v}.

\begin{figure}[h]
   \begin{minipage}[c]{.32\linewidth}
      \includegraphics[width=1\textwidth,height=0.8\textwidth]{images/v_3d}
      %\caption{Profil 3D du potentiel}
      \label{f v_3d}
   \end{minipage} \hfill
   \begin{minipage}[c]{.32\linewidth}
      \includegraphics[width=1\textwidth,height=0.8\textwidth]{images/v_xy}
      %\caption{Vue $xOy$ du potentiel}
      \label{f v_xy}
   \end{minipage} \hfill
   \begin{minipage}[c]{.32\linewidth}
      \includegraphics[width=1\textwidth,height=0.8\textwidth]{images/v_yz}
      %\caption{Vue $yOz$ du potentiel}
      \label{f v_yz}
   \end{minipage}
   \caption{Différentes vues du potentiel}
   \label{f v}
\end{figure}
On voit sur le tracé de gauche que dans ce cas là on observe bien des effets de bord. Au centre du fil ($z=0$) la forme du potentiel est relativement proche de celle qu'on a pu trouver dans le cadre du modèle où le tube était considéré infini.



%\begin{figure}[h]
%\centering
%\includegraphics[width=0.5\textwidth,height=0.4\textwidth]{images/v_3d}
%\caption{Profil 3D du potentiel}
%\label{f v_3d}
%\end{figure}

%\begin{figure}[h]
%\centering
%\includegraphics[width=0.5\textwidth,height=0.4\textwidth]{images/v_xy}
%\caption{Vue $xOy$ du potentiel}
%\label{f v_xy}
%\end{figure}

%\begin{figure}[h]
%\centering
%\includegraphics[width=0.5\textwidth,height=0.4\textwidth]{images/v_yz}
%\caption{Vue $yOz$ du potentiel}
%\label{f v_yz}
%\end{figure}


\subsection{Calcul des trajectoires électroniques dans le potentiel MEF}


Une fois que l'on a un programme qui donne la trajectoire de l'électron dans un potentiel et un programme qui détermine le potentiel dans le tube en tenant compte des effets de bord, on peut combiner ces deux programmes pour obtenir la trajectoire de l'électron dans un potentiel déterminé numériquement. La fonction \verb|pick.m| permet de récupérer le potentiel en un point à l'intérieur du cylindre, ce qui va nous permettre de calculer la trajectoire de l'électron dans le tube.

On utilise la méthode RK4 (vue au \ref{subsection:lol}) pour résoudre le système différentiel, sauf qu'ici les valeurs du potentiel et du gradient de potentiel ne sont plus données par une formule analytique mais par un programme qui les calcules à partir de la résolutions éléments finis. Un tracé de la trajectoire vu en $xy$ est présenté sur la figure \ref{fig:elie_encule}, les conditions initiales et les paramètres de calculs sont les mêmes que ceux utilisés au \ref{subsection:lol}.

\begin{figure}[h]
\centering
\includegraphics[height=6cm]{images/t_num_xy.png}
\caption{Trajectoire d'un électron projetée sur un plan $z=\mathrm{cste}$}
\label{fig:elie_encule}
\end{figure}

Dans ce cas, la trajectoire est proche d'une hélice, on voit par contre les effets dus aux inhomogénéités du potentiel car l'hélice n'est plus <<circulaire>>.



\subsection{Energie potentielle effective}

En plus de la connaissance de la trajectoire de l'électron il peut être intéressant de réfléchir à la notion d'énergie potentielle effective. L'énergie totale de l'électron s'écrit de la manière suivante:
\[
E = \frac 1 2 mv^2 - eV
\]
Soit:
\[
E = \frac 1 2 m \dot{r}^2 + \frac 1 2 m \dot{z}^2 + \frac 1 2 m r^2 \dot{\theta}^2 - e V
\]
Comme dans le cas de l'étude d'un mouvement planétaire, on va définir l'énergie potentielle effective comme l'énergie totale à laquelle on enlève les termes en $\dot{r}^2$ et en $\dot{z}^2$. On peut donc définir une énergie potentielle effective de la forme:
\[
E_{\mathrm{eff}} = \frac 1 2 m r^2 \dot{\theta}^2 -e V
\]
On calcule numériquement cette grandeur en chaque point de la trajectoire. On trace ensuite l'énergie potentielle effective en fonction des deux coordonnées $r$ et $z$, pour obtenir un résultat comme ceux des figures \ref{f Eeff_3d} et \ref{f Eeff}. Cette cuvette est à comparer à l'énergie totale (qui est une constante) pour connaître les limites spatiales du mouvement.

\begin{figure}[h]
   \begin{minipage}[c]{.49\linewidth}
      \includegraphics[width=0.99\textwidth,height=0.8\textwidth]{images/Eeff_3d}
      %\caption{Profil 3D du potentiel}
      \label{f Eeff_3d}
   \end{minipage}
   \begin{minipage}[c]{.49\linewidth}
      \includegraphics[width=0.99\textwidth,height=0.8\textwidth]{images/Eeff_rEeff}
      %\caption{Vue $xOy$ du potentiel}
      \label{f Eeff_xy}
   \end{minipage}
   \caption{Différentes vues du potentiel effectif}
   \label{f Eeff}
\end{figure}

On peut faire plusieurs commentaires sur les deux courbes des figures \ref{f Eeff}. Elles présentent bien un aspect de <<cuvette>> mais sont particulièrement bruitées. Nous avons procédés à plusieurs essais pour tenter d'obtenir de meilleurs résultats. Nous avons d'abord augmenté le nombre de plus proches voisins utilisé dans le fichier \verb|pick.m|, sans résultat significatif. Nous avons aussi tenté de diminuer la valeur du pas d'intégration (qui était dans ce cas $1.10^{-12}$), comme déjà indiqué une telle diminution entraîne des calculs interminables.


\begin{figure}[h]
\centering
\includegraphics[height=6cm]{images/Eeff_3d_ci_2.png}
\caption{Energie potentielle effective}
\label{f Eeff}
\end{figure}

\section{discussion}



\section{Conclusion}
Afin de résoudre ce problème complexe nous avons procédé de manière incrémentale. Nous avons d'abord simulé la trajectoire d'un électron dans un potentiel dont la solution analytique est connue grâce à la méthode RK4. Nous avons ensuite simulé le potentiel créé par le fil dans le tube en tenant compte des effets de bord par la méthode des éléments finis. Une fois ces deux choses réalisées nous avons pû fusioner les deux programmes afin d'obtenir la trajectoire de l'électron dans le tube en prenant en compte les effets de bord.  Nous avons obtenu des résultats satisfaisants donnant la trajectoire de l'électron dans le tube.


Cependant des problèmes apparaissent lorsque l'on utilise un maillage plus fin, en effet on observe une divergence de la trajectoire de l'électron ce qui pose problème puisque notre résolution numérique ne devrait logiquement pas dépendre du maillage. Après plusieurs simulations nous avons conclu que le problème venait sûrement des conditions initiales ou de la fonction \emph{pick} qui donne le potentiel calculé numériquement en un point. Nous avons tenté plusieurs manières de résoudre ce problème. Nous avons d'abord utilisé plusieurs versions de la fonction \emph{pick} puis testé différents jeux de conditions initiales, sans résultat.
























\end{document}